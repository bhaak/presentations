\documentclass[handout]{beamer}

\usepackage[english]{babel}   
\usepackage[latin1]{inputenc}
\usepackage[T1]{fontenc}

% http://www.hartwork.org/beamer-theme-matrix/
\usepackage{beamerthemesplit}
%\useoutertheme{shadow}


\usepackage{colortbl}

\usepackage{wasysym}
\usepackage{csquotes}
\usepackage{listings}

\newcommand{\q}[1]{\enquote{#1}}%

\title{Junethack - NetHack Cross-Variant Summer Tournament}
\author{Patric Mueller}
%\date{\today}
\date{3 Jun 2011}

\begin{document}

\begin{frame}
\titlepage
\end{frame}

%\section[outline]{}
%\frame{\tableofcontents}

%\section{introduction}
\begin{frame}
\frametitle{Who's that guy?}
My name's Patric Mueller \href{mailto:bhaak@gmx.net}{<bhaak@gmx.net>} \pause

I'm the developer of \pause

  \begin{itemize}[<+->]
    \item UnNetHack\footnote{\onslide<3->{\url{http://apps.sf.net/trac/unnethack/}}} - a variant of NetHack "how NetHack would look like if it were still in development"
    \item NetHack--De\footnote{\onslide<4->{\url{http://nethack-de.sf.net/}}} - an almost complete translation of NetHack into German.
    \item Junethack\footnote{\onslide<5->{\url{http://junethack.rawrnix.com//}}} - The NetHack Cross-Variant Summer Tournament
  \end{itemize}
\end{frame}

\begin{frame}
\frametitle{Junethack}
Junethack is
  \begin{itemize}
    \item an online tournament
    \item for NetHack and several of its forks
    \item using existing public servers
  \end{itemize}
\end{frame}

%\section{NetHack tournaments}
\begin{frame}
\frametitle{pre-Junethack situation}
  Existing NetHack tournaments before Junethack:\pause
  \begin{itemize}[<+->]
    \item "the June tournament" conducted by Eidolos in 2006 and 2007 on \href{http://nethack.alt.org/}{NAO}
    \item \href{http://nethack.devnull.net/}{the /dev/null/nethack tournament} (usually just "devnull") running since 1999
  \end{itemize}
\end{frame}

\begin{frame}
\frametitle{June tournament on NAO}
  The "June tournament" was only hold twice AFAIK:\pause
  \begin{itemize}[<+->]
    \item June 2006: 18,748 games, 188 ascensions
    \item June 2007: 14,524 games, 349 ascensions
  \end{itemize}\pause
  \vfill
  BTW, \href{http://alt.org/nethack/perday.html}{NAO statistics} are somewhat constant:\\
  ~April 2011 \textasciitilde 18,000 games, 171 ascensions
  \vfill
\end{frame}

\begin{frame}
\frametitle{/dev/null/nethack}
  devnull has been holding an annual NetHack Tournament, beginning at midnight on Halloween, since 1999.\pause
  \begin{itemize}[<+->]
    \item 2008: 22,812 games by 1305 players (only 58 players ascended!)
  \end{itemize}\pause
  \vfill
  devnull has been a key influence on the design of Junethack.
  \vfill
\end{frame}

\begin{frame}
\frametitle{/dev/null/nethack (continued)}
  devnull features\pause
  \begin{itemize}[<+->]
    \item public servers setup only for the tournament
    \item only vanilla NetHack (no GUI improvements, no bugfixes)
    \item lots of trophies
    \item clan scores
    \item non-NetHack challenges (triggered e.g. by wishing, digging, etc.)
    \begin{itemize}[<+->]
      \item in-game challenges (e.g. Grue, Pac-Man level)
      \item out-of-game challenges (ZapM and KoL)
    \end{itemize}
    \item starting at midnight Halloween Pacific Standard Time (PST)
  \end{itemize}
\end{frame}

\begin{frame}
\frametitle{Junethack design}
  With experiences from devnull and some more feature request from IRC, these were the ideas and design goals for Junethack\pause
  \begin{itemize}[<+->]
    \item using existing public servers
    \item no games being played on the tournament server (only hosting Junethack website)
    \item offering vanilla NetHack \textit{and} forks and a bonus game (NetHack 1.3d from 1987)
    \item encouraging playing multiple variants
    \item lots of trophies
    \item clan scores
    \item more appealing to newbies or non-ascenders (hopefully)
    \item people should be able to register at any time and all their games from the tournament period should be counted
    \item starting at midnight Coordinated Universal Time (UTC) (plus no timezone based trophies)
  \end{itemize}
\end{frame}

% technology slide
\begin{frame}
\frametitle{Junethack implementation (aka buzz word bingo)}
  Junethack is\pause
  \begin{itemize}[<+->]
    \item implemented in Ruby
    \item using the Sinatra web framework
    \item using datamapper as ORM
    \item using Sqlite3 as database backend
    \item source code available from GitHub \url{https://github.com/junethack/Junethack}
  \end{itemize}
\end{frame}

% technology slide
\begin{frame}
\frametitle{Junethack implementation (continued)}
  For a public server to take part in Junethack,\\it has to fullfill 2 conditions:\pause
  \begin{itemize}[<+->]
    \item having a downloadable xlogfile log (standard patch)
    \begin{itemize}[<+->]
      \item so there is a standard way of reading machine-translatable data about played games and achievements
      \item the Junethack server is polling every 15 minute for changed xlogfiles, new games are added to the database regardless if the player is registered or not
    \end{itemize}
    \item letting the player write a custom line of text that can be downloaded somehow (usually this is written as comment into the option file)
    \begin{itemize}[<+->]
      \item for verifiying that Junethack user A is user B on a given public server
      \item unfortunately, the user has to do this once for every public server
    \end{itemize}
  \end{itemize}
\end{frame}

%\section{Junethack 2011}
\begin{frame}
\frametitle{Junethack Timeline}
  \pause
  \begin{itemize}[<+->]
    \item late 2010, early 2011: ideas for a new NetHack tournament coming up on IRC
    \item early 2011: planning for achievements and trophies, some web mockups
    \item May 2011: actual code gets written
    \item June 2011: lots of code gets written. At that time, I get involved.
    \item June 2011: discussions about the name flare up -> decision: as long as it starts at least in \textbf{Ju}ly, it's fine
    \item 17th July 2011: start of the tournament (not all ready, not all trophies implemented, but for preventing overlap with Crawl tournament and vicinity to devnull)
    \item 13th August 2011: last clan trophy implemented (1 day before tournament ends)
    \item 14th August 2011: tournament ends
    \item 1st July 2012: second Junethack tournament starts
  \end{itemize}
\end{frame}

% issues during the tournament
%\section{Junethack 2011}
\begin{frame}
\frametitle{Events and issues during Junethack 2011}
  \pause
  \begin{itemize}[<+->]
    \item a week into the tournament, we got slashdotted: \url{http://games.slashdot.org/story/11/07/22/2338215/first-nethack-cross-variant-summer-tournament}
    \item some performance issues
    \begin{itemize}[<+->]
      \item because of slashdot
      \item because of missing caching (missing because we started "early")
      \item because of startscummer and thread issues
      \begin{itemize}[<+->]
        \item parsing start scummed games took longer than 15 minutes, concurrent running parsing processes
        \item lots of caching depends on date of last game played by a registered user
        \item opening a start scummer user page would hog the CPU (all played games are shown)
      \end{itemize}
    \end{itemize}
    \item For recalculating trophies and clan score, we had to re-import several times all games because of code-fixes, code changes, etc.
  \end{itemize}
\end{frame}

% start scummer badness 
\begin{frame}
\frametitle{Junethack 2011 - a start scummer's Eldorado}
  So, how bad was start scumming in Junethack 2011?\pause\\
  Very, very bad.\pause\\
  A game is considered to be start scummed if its turn count < 10 and the death reason is 'escaped' or 'quit' (the same definition as NAO uses).\pause\\
  There were \textbf{125,252} of such games and only 18,079 regular games.\pause\\
  I only tested the database with 50,000 games (more than 2 months of normal NAO activity). :-)\pause\\
  Two players accounted for most of those games.\pause\\
  One for 76,546 and the other for 22,703 games.\pause\\
  Perfomance issues were resolved by not writing start scummed games into the main games table.
\end{frame}

% statistics


% lesson learnt
% no thoughtful preparation will help you against NetHack players
%  think about obsessive players in your community "What is the worst X could do?"
% recalculate we had to reimport all the games several times because of bugs issues with parsing/code changes/etc. -> not disruptive to the player
% people are really forgiving if you aren't disrupting their games

\begin{frame}
\frametitle{Junethack 2012}
  Not much was changed for this year, small bugfixing and polishing the website:\pause
  \begin{itemize}[<+->]
    \item most of the streak trophies have been removed
    \item new forks: GruntHack and NetHack4
    \item new clan trophy: most trophy combinations
    \item the tournament is in June (and will be in the future)
    \item no bonus game (removed NetHack 1.3a)
    \item The tournament web site look has been slightly updated and bugs have been fixed
  \end{itemize}
\end{frame}

\begin{frame}
\frametitle{Future ideas?}
  \begin{itemize}[<+->]
    \item replacing clan "most points" with "least points"?
    \item adding more trophies? (not easy of xlogfile)
    \item allowing translated versions of NetHack? (there are Japanese, Spanish and
German versions)
    \item ...
  \end{itemize}
\end{frame}

\end{document}
